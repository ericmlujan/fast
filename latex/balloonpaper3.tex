% mn2esample.tex
%
% v2.1 released 22nd May 2002 (G. Hutton)
%
% The mnsample.tex file has been amended to highlight
% the proper use of LaTeX2e code with the class file
% and using natbib cross-referencing. These changes
% do not reflect the original paper by A. V. Raveendran.
%
% Previous versions of this sample document were
% compatible with the LaTeX 2.09 style file mn.sty
% v1.2 released 5th September 1994 (M. Reed)
% v1.1 released 18th July 1994
% v1.0 released 28th January 1994

\documentclass[useAMS,usenatbib]{mn2e}
\usepackage{graphicx,subfigure}
\usepackage{bm}
\usepackage{longtable}
\usepackage{amsmath,amssymb,graphicx}
\usepackage{rotating}
\usepackage{color}
\usepackage{comment}

% If your system does not have the AMS fonts version 2.0 installed, then
% remove the useAMS option.
%
% useAMS allows you to obtain upright Greek characters.
% e.g. \umu, \upi etc.  See the section on "Upright Greek characters" in
% this guide for further information.
%
% If you are using AMS 2.0 fonts, bold math letters/symbols are available
% at a larger range of sizes for NFSS release 1 and 2 (using \boldmath or
% preferably \bmath).
%
% The usenatbib command allows the use of Patrick Daly's natbib.sty for
% cross-referencing.
%
% If you wish to typeset the paper in Times font (if you do not have the
% PostScript Type 1 Computer Modern fonts you will need to do this to get
% smoother fonts in a PDF file) then uncomment the next line
% \usepackage{Times}

%%%%% AUTHORS - PLACE YOUR OWN MACROS HERE %%%%%

\newcommand{\f}[2]{\frac{#1}{#2}}
%%%%%%%%%%%%%%%%%%%%%%%%%%%%%%%%%%%%%%%%%%%%%%%%

\title[]{High Altitude Ballooning: A Comprehensive Review}
\author[Amanda Maxham, etc.]{Amanda Maxham\thanks{E-mail:
maxham@physics.unlv.edu (AM); lujane@unlv.nevada.edu (EL); maxham@physics.unlv.edu (IR); maxham@physics.unlv.edu (HOL) } Eric Lujan, Ian Rabago and Han Oel Lee\\
Department of Physics and Astronomy, University of Nevada, Las Vegas, Las Vegas, NV, 89124, U.S.A.}
\begin{document}

\date{Accepted . Received ; }

\pagerange{\pageref{firstpage}--\pageref{lastpage}} \pubyear{2011}

\maketitle

\label{firstpage}

\begin{abstract}
The abstract will go here. It's a summary of the below paper.
\end{abstract}

\section{Introduction}
With the recent interest in high altitude ballooning from both University, amateur radio and non-specialists, many are seeking out more information on the science behind high altitude ballooning. In this paper, we review some of the science, equipment and calculations behind how to have a successful launch and feature a few interesting student projects.

\section[]{Terminal Velocity of Ascent}
A launched balloon is acted on by three forces in the z-direction, the upwards buoyant force, $F_{B}$,  %of the balloon is caused by the buoyancy of
%He gas and is affected by the mass of air that is displaced by the He. The forces of gravity ($F_g$) and drag ($F_d$) act synergistically in a
%downward direction and resist $F_{He}$.the upwards buoyant force, $F_B$, 
the downward force of the combined weight of the balloon, gas inside the balloon and the payloads attached, $F_g$, and the drag force, $F_d$ (see figure xx).  After the balloon is launched, the over balance of the upward buoyant force causes the balloon to accelerate upwards. Figure 07 shows an altitude vs. time plot for a high altitude weather balloon launch which is typical of these types of flights. The downward drag force quickly overcomes the initial upward force, causing the balloon to reach terminal velocity and remains at a constant velocity throughout the flight despite the changing surface area of the balloon and air density both of which can effect drag.


 A given mass of He displaces 7.125 times its mass in air, therefore, the upwards force of the helium can be calculated by 

\begin{equation}
F_{He} = (7.125m_{He} - m_{He}) \cdot 9.81
\end{equation}
where $m_{He}$ is the mass of helium in the balloon in kilograms. 

As the balloon accelerates, the values of the three forces can be combined to a net upwards force that can be described as follows:

\begin{equation}
\Sigma F= F_{He} - F_g - F_d
\end{equation}


These forces are at equilibrium when the balloon reaches terminal velocity, which results in $\Sigma F$ = 0.

\subsection[]{Time}
The approximate time it takes to reach terminal velocity can be computed if terminal velocity and the acceleration are known by using the following formula where $a$ represents the initial acceleration of the balloon:

\begin{equation}
V = \frac{1}{2}a\cdot{}t^{2}
\end{equation}



In order to determine the initial acceleration of the balloon, we can apply Newton's second law:

\begin{equation}
\Sigma F = ma
\end{equation}

\subsection[]{Experimental application}
In our launch, we were able to measure the rate of ascent at terminal velocity due to the presence of GPS receivers on the balloon. We found that the ascent rate was approximately 5.59 m/s. In addition, we estimate the mass of helium that went into the balloon as 1.63 kg. By applying Equation 1, we computed that the value of $F_{He}$ is 97.94 N.

\begin{equation}
F_{He} = 9.81 (7.125 \cdot 1.63  - 1.63) = 97.94
\end{equation}

Since we know that the total mass of the balloon and its payloads is 7.736 kg, $F_g$ has a value of 75.89 N.

\begin{equation}
F_g = 7.736 \cdot 9.81 = 75.89
\end{equation}

To find $F_d$, we can apply the drag equation:
\begin{equation}
F_d = C A(\frac{\rho \cdot V^2}{2})
\end{equation}

Where $\rho$ is the density of the air (1.2kg/m$^3$), $V$ is the balloon's velocity, $A$ is the cross-section area of the balloon (1.49$\pi$ m$^2$) , and $C$ is the drag coefficient of the balloon (~0.25 in our case).\\



By applying Equation 2, we find that the initial net upward force on the balloon is 22.05 N.

\begin{equation}
\Sigma F = 97.94 - 75.89 - 0 = 22.05
\end{equation}

The initial acceleration of our balloon can be determined by applying Newton's second law to our known values:

\begin{equation}
22.05 = 7.736 \cdot a
\end{equation}

This results in an initial acceleration value of 2.85 m/s$^2$\\

By taking these values and implementing them in Equation 3, we can predict the approximate time that it takes for the balloon to accelerate to terminal velocity.

\begin{equation}
5.59 = \frac{1}{2} \cdot  t^2
\end{equation}

We can estimate that our balloon will reach terminal velocity (5.59 m/s) after 3.34 seconds of acceleration.

We can find the forces present at terminal velocity by re-applying the above equations:

\begin{equation}
F_{He} = 97.94
\end{equation}

\begin{equation}
F_g = 7.736 \cdot 9.81 = 75.89
\end{equation}

\begin{equation}
F_d = 0.25 \cdot 1.49 \pi (\frac{1.2\cdot(5.59)^2}{2}) = 21.94
\end{equation}

At terminal velocity, $\Sigma F$ is approxmiately equal to 0.1 N.

\begin{equation}
\Sigma F = F_{He} - F_g - F_d = 97.94 - 75.89 - 21.94 = 0.11
\end{equation}


Please note that the preceding methods are for estimation purposes only. Calculus (particluarly deriviatives) can be used to accurately predict the time a balloon reaches terminal velocity, but these equations have not yet been adapted for use in a positive vector field.

\section[]{The composition of particle sphere visible upon balloon burst}

This section will be furthered in experimental trials and will be published independently.\\

See notices of deletion in Appendix A.

\section[]{The effect of wind vectors on the ascent and descent paths of
high-altitude balloons}
Figure 1 shows the path of a high altitude balloon flight from the Flying Apple Space Technologies (FAST) group associated with the University of Nevada, Las Vegas (UNLV). This trajectory, like others, shows the interesting property that the launch, burst and landing points are all found to lie on a single straight line. Further, the rising part of the flight path, between launch and burst and the descending part of the curve between burst and landing appear to be mirror images of each other, with the descending part being a foreshortened image.

Wind speed and direction as a function of altitude can be described as a vector field that applies a force to a balloon or parachute and cause displacement along the path. The relationship between the ascent rate, descent rate, and this vector field gives rise to this flight path pattern and is described mathematically below.

\subsection[]{Vector equations}
As a balloon ascends through the atmosphere, it is affected by multiple wind vectors. The balloon quickly reaches terminal velocity (within x seconds, see figure xx) and rises with a constant velocity until burst. After burst, the payloads on the flight string also quickly reach terminal velocity, whether carried down on a parachute or not.

The wind exerts three forces on the balloon or parachute $F_x$, $F_y$ and $F_z$ at any point, with the three dimensional plane defined such that the positive z-axis points normal to the surface of the Earth. Assuming that the wind force in the z-direction, $F_z$ is negligible, the vector wind field in the x-y plane causes the balloon to move laterally as it travels upwards or downwards through this field. Assuming that the balloon and payloads travel through the same vector field on the way up as the way down, this causes the three points,
launch point, burst, and landing, to lie along the same line. The ascent path appears to be rotated $180 ^\circ$ about the burst point and scaled by a factor to create the descent path.



%\begin{figure}
%\includegraphics[scale=0.3]{trajectory2.eps}
%\caption{A hypothetical trajectory.}
%\label{traj}
%\end{figure}
If $\kappa$ is ratio of the ascent velocity of the balloon in the z-direction, $v_+$,  and the descent velocity of the payloads in the negative z-direction is $v_-$, then,

\begin{equation}
\kappa = \f{v_+}{v_-} = \f{d_-}{d_+} = \f{t_-}{t_+}
\label{Mandm}
\end{equation}

% --------------------------------------------------------------
% Note by Eric Lujan:
% This is a section for our reference only; please remove before
% final draft is prepared for publication.
% --------------------------------------------------------------

\subsection{List of Equations}
These equations are from the previous LaTeX file, and listed here for reference as well as for future insertion.

\begin{equation}
D_{t} = D_{a}+D_{d}
\end{equation}

\begin{equation}
D_{d} = K \cdot D_{a}
\end{equation}

\begin{equation}
K = \frac{r_{a}}{r_{d}}
\end{equation}

\begin{equation}
D_{d} = (\frac{r_{a}}{r_{d}}) \cdot D_{a}
\end{equation}

\begin{equation}
D = D_{a} + (\frac{r_{a}}{r_{d}}) \cdot D_{a}
\end{equation}

\begin{equation}
t_{l} = t_{a} + t_{d}
\end{equation}
\begin{equation}
t_{d} = K \cdot t_{a} = \frac{r_{a}}{r_{d}} \cdot t_{a}
\end{equation}
\begin{equation}
t_{l} = t_{a} + \frac{r_{a}}{r_{d}} \cdot t_{a}
\end{equation}

These equations are listed in the order that they appear in the original document.

% --------------------------------------------------------------
% Note by Eric Lujan:
% This section has been commented out to denote pending
% deletion. Awaiting approval to redact stabilization section.
% Amanda: please email me or submit an issue on GitHub.
% --------------------------------------------------------------

\begin{comment}

\section[]{Calculating inertia of stabilization rods (Pasted Verbatim: Not edited yet)}
Up in the atmosphere, payloads attached to an ascending weather balloon can spin rapidly due to changing and turbulent wind streams.  This causes problems for camera packages, as the images or videos are disorienting and blurry.  For these kinds of packages, stabilization rods are useful for keeping the string from kinking and the cameras spinning slowly.  However, there are many rods to use, such as a tapered rod, like a fishing pole, or a uniform rod such as a dowel.  The mathematics and effectiveness of these is shown below.
\subsection{Uniform Rods}
The rods that were used for the FAST-6 and FAST-7 weather balloons were straight stabilization rods.  They consisted of 3 light fiberglass rods attached by the center onto either end of the camera payloads.  They ended up reducing the spin of the cameras long enough to get amazing pictures.  Of course, the more mass, the slower the packages spin.  The calculations to find the factor of slowing are as follows:

Assuming the package experiences the same torque from a wind current or the balloon pulling on it, we will see the difference of angular acceleration, $\alpha$.  Because $\tau$ = I $\alpha$, we must find the moment of inertia to find the reduction of angular acceleration.  For our purposes, the original moment of inertia I will be the package with no rods.
In this example we will use a single rod attached to the package by the end.  Assuming the rod has a uniform density $\rho$, a mass m, and a length L, we can use those to find the moment of inertia I.

\begin{equation}
I = \int r^{2} \: dm
\end{equation}

We must integrate the mass along the entire length of the rod, so r becomes L and we integrate from 0 to L.

\begin{equation}
I = \int_{0}^{L} L^{2} \: dm
\end{equation}

Unless the rod used is of significant thickness, the thinckness of the rod can be ignored. dm can then be expressed in terms of dL using density:

\begin{equation}
$$ $ \rho = \frac{m}{L} = \frac{dm}{dL} $ $$
\end{equation}
\begin{equation}
$$ $ dm = \rho \cdot dL $ $$
\end{equation}

Now we substitute back:

\begin{equation}
$$ $ I = \int _{0}^{L} L^{2} \cdot \rho \: dL = \rho \int _{0}^{L} L^{2} dL  $ $$
\end{equation}

\begin{equation}
I = \rho \left[ \frac{1}{3} L^{3} \right] _{0}^{L} = \frac{\rho L^{3}}{3}
\end{equation}

As density was defined above as mass per unit length, the equation becomes:

\begin{equation}
I = \frac{mL^{3}}{3L} = \frac{mL^{2}}{3}
\end{equation}

This is the inertia of one rod.  For the total inertia of n rods, the equation becomes:

\begin{equation}
I = \frac{nmL^{2}}{3}
\end{equation}

\subsection{Tapered Rods}

Our alternate idea for a stabilization rod was to be a fishing rod, tied to the package at the rod's center of balance.  However, a fishing rod is telescopic, and shrinks towards the end.  This adds another level in complexity when finding the moment of inertia.

We will use the same idea as the previous problem, but now we must solve for two different areas:  the widening part to the left of the center of mass (referred to as C.O.M. later) and the tapering  part to the right.  However, we need to find the center of mass first.  The distance to center of mass R from the left of an object is given by the equation:

\begin{equation}
R = \frac{1}{M} \int r \: dm = \frac{1}{M} \int \rho L \: dV
\end{equation}

We will be using the second version of the equation to deal with this problem.  We integrate along the length of the rod, so dL is needed.  To do this, we take a look at an infinitesimally small slice of the rod, which becomes a cylinder with height dL.

Picture 1

\begin{equation}
V = \pi r^{2} h
\end{equation}

\begin{equation}
dV = \pi r^{2} dL
\end{equation}

However, the radius in this equation changes as you move across the length, and the rate of change is related to the maximum radius and the length of the rod.  Take a look at the picture below:

Picture 2

If the rod tapers off linearly, then the radius (which will be referred to as r') at any given point is equal to the slope of the rod times the distance from the maximum radius (which will be called L').  The slope of the rod can be calculated by dividing the maximum radius (called $r_{max}$) by the total length of the rod (called $L_{max}$).  This can be expressed mathematically as follows:

\begin{equation}
r' = \frac{r_{max}}{L_{max}} \cdot L'
\end{equation}

If the rod tapers but doesn't actually come to a point by the end of a rod, the slope of the rod is calculated by dividing the radius decrease by the length.

Plugging this into the volume equation earlier, we get:

\begin{equation}
dV = \pi r' \, ^{2} dL = \pi \left( \frac{r_{max}}{L_{max}} \right) ^{2} L' \, ^{2}
\end{equation}

Now we plug this into the C.O.M. equation and solve.

\begin{equation}
R = \frac{1}{M} \int \rho L \: dV = \frac{1}{M} \int \rho L  \left( \pi \frac{r_{max}}{L_{max}} ^{2} L' \, ^{2}  \right) dL
\end{equation}

We integrate along the entire rod.  It's important to note that L in the original equation is the same as L’ in the radius equation.  We can combine these to solve the integral.  I'll also be turning M into m for consistency's sake.  Density is $\frac{m}{V}$ here, so I'll substitute that in as well.

\begin{equation}
R = \frac{ \rho \pi r_{max} \: ^{2}}{m L_{max} \: ^{2}} \int_{0}^{L} L^{3} dL
\end{equation}

\begin{equation}
R = \frac{ \rho \pi r_{max} \: ^{2}}{m L_{max} \: ^{2}} \left[ \frac{1}{4} L^{4} \right] _{0} ^{L}
\end{equation}

\begin{equation}
R = \frac{ \pi m r_{max} \: ^{2} L_{max} \: ^{4} }{4 m V L_{max} \: ^{2}} = \frac{\pi r_{max} \: ^{2} L_{max} \: ^{2}}{4 V}
\end{equation}

This can be simplified further by taking the volume of a tapered rod as a cone with base radius $r_{max}$ :

\begin{equation}
V = \frac{1}{3} \pi r^{2} h = \frac{1}{3} \pi r_{max} \: ^{2} L
\end{equation}

\begin{equation}
R = \frac{\pi r_{max} \: ^{2} L_{max} \: ^{2}}{4 \left( \frac{1}{3} \pi r_{max} \: ^{2} L _{max} \right) } = \frac{3}{4} L_{max}
\end{equation}

I wil call this value R for the time being.  It helps clear up the next few steps.  Now we need to use this distance to find the two integrals needed for the inertia calculations.

Picture 3

Using this diagram, we see that the bounds of the two integrals will be from R to L=0 (which is also –R) and R to L=$L_{max}$ (which is the distance $L_{max}$ – R).  Because R is on the C.O.M. of the package, we can call it x=0.  We can now modify the original inertia equation for these purposes.

\begin{equation}
I = \int r^{2} \: dm
\end{equation}

Instead of using dL as before, we replace it with dV due to the changing shape of the rod.  dV will be replaced with the expression found earlier.

\begin{equation}
I = \int_{0} ^{L} L^{2} \cdot \rho \: dV
\end{equation}

\begin{equation}
I = \int_{0} ^{L} L^{2} \cdot \rho \pi \left( \frac{r_{max}}{L_{max}} \right) ^{2} L' \, ^{2} dL
\end{equation}

\begin{equation}
I = \rho \pi \left( \frac{r_{max}}{L_{max}} \right) ^{2} \int_{0} ^{L} L^{4} \: dL
\end{equation}

However, the integral must be broken into the two parts explained earlier.  Pulling out the constants in the equation, we get:

\begin{equation}
I = \rho \pi \frac{r_{max} \, ^{2}}{L_{max} \, ^{2}} \left[ \left( \int_{-R}^{0} L^{4} \: dL \right) + \left( \int_{0} ^{L-R_{max}} L^{4} \: dL \right)  \right]
\end{equation}

Now for the integration and solving.

\begin{equation}
I = \rho \pi \frac{r_{max} \, ^{2}}{L_{max} \, ^{2}} \left\lbrace \left( \left[ \frac{1}{5} L^{5} \right]_{-R} ^{0} \right) + \left( \left[ \frac{1}{5} L^{5} \right]_{0} ^{L-R_{max}} \right)  \right\rbrace
\end{equation}

\begin{equation}
I = \rho \pi \frac{r_{max} \, ^{2}}{L_{max} \, ^{2}} \left[ \left( \frac{1}{5} R^{5} \right) + \left( \frac{1}{5} \left( L-R \right)^{5} \right)  \right]
\end{equation}

\begin{equation}
I = \frac{ \rho \pi r_{max} \, ^{2}}{5 L_{max} \, ^{2}} \left[ R^{5} + \left( L-R \right)^{5} \right]
\end{equation}

Substituting in R, we get:

\begin{equation}
I = \frac{ \rho \pi r_{max} \, ^{2}}{5 L_{max} \, ^{2}} \left[ \left( \frac{3}{4} L \right) ^{5} + \left( L- \frac{3}{4} L  \right)^{5} \right]
\end{equation}

\begin{equation}
I = \frac{ \rho \pi r_{max} \, ^{2}}{5 L_{max} \, ^{2}} \left[ \left( \frac{243}{1024} L^{5} \right) + \left( \frac{1}{1024} L^{5} \right) \right]
\end{equation}

\begin{equation}
I = \frac{ \rho \pi r_{max} \, ^{2}}{5 L_{max} \, ^{2}} \left( \frac{61}{256} L^{5} \right) = \frac{61 \rho \pi r_{max} \, ^{2} L^{3}}{1280}
\end{equation}

Substituting in $\frac{m}{V}$ for $\rho$ as before:

\begin{equation}
I = \frac{61 m \pi r_{max} \, ^{2} L^{3}}{1280\left( \frac{1}{3} \pi r_{max} \: ^{2} L \right) } = \frac{183 m L^{2}}{1280}
\end{equation}

Now the inertia is in terms of the mass and length of the rod, which is easy for someone to measure.  For n rods attached in this method, the equation becomes:

\begin{equation}
I = \frac{183 n m L^{2}}{1280}
\end{equation}

\end{comment}

% --------------------------------------------------------------
% Note by Eric Lujan:
% End of content pending deletion.
% --------------------------------------------------------------

\section{Latex Backpressure and Lift}

\subsection{A Conceptual }

Backpressure (also known as "membrane pressure") is the inwards pressure that the elasticity of the balloon places on the gases contained therein. It was determined that backpressure changes over the course of the flight, due to differing balloon radius and atmospheric pressures, and can have a significant effect on the amount of lift that the gas provides. The pressure inside the balloon $P_t$ can be represented mathematically 

\label{lastpage}

\end{document}
